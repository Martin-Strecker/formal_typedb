\section{Open Issues}\label{sec:open_issues}


\begin{issue}[Thing type]\label{issue:thing_type}
The grammar contains a type \texttt{thing}, but it is not clear in which
  contexts it can be used. A declaration of the form \texttt{foothing sub
    thing;} leads to an error (\texttt{[CNX07] TypeDB Connection: Transaction
    was closed due to: INTERNAL: [TYW35] Invalid Type Write: The type
    'foothing' cannot be defined, as the provided supertype 'thing' is not a
    valid thing type..}).\crash{Transaction closed}
\end{issue}

\begin{issue}[Declaration of attributes]\label{issue:decl_attribute}
For a declaration of an attribute, it is apparently necessary to write
something like:
\begin{verbatim}
serial_num sub attribute,
    value long;
\end{verbatim}
The declaration \texttt{sub attribute} looks redundant as only attributes can
have a value declaration. So why not simply write the following for an
attribute declaraion?
\begin{verbatim}
serial_num value long;
\end{verbatim}

Also, it is not clear under which conditions an attribute can be subtyped. The
  following is illegal:
  \begin{verbatim}
serial_num sub attribute,
    value long;
serial_subnum sub serial_num;
  \end{verbatim}
  
  and leads to the error message \texttt{[TYW16] Invalid Type Write: The
    attribute type 'serial\_num' cannot be subtyped as it is not abstract}.
  Even leads to a bad crash of the TypeDB server (no more commit possible,
  server has to be restarted). \crash{Crash requiring server restart}
\end{issue}


%%% Local Variables:
%%% mode: latex
%%% TeX-master: "main"
%%% End:
