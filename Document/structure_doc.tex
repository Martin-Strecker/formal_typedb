\section{Structure of the Document}

In this document, the authors try to gain a better understanding of the TypeDB
system, in particular the relation of the type system and its query and update
language.

This document is currently in a \emph{very preliminary form} but is likely to
evolve in the following months. Its \textbf{structure} is:

\begin{itemize}
\item As the documeent evolves, a rigorous, but still informal presentation is
  planned in the current chapter.
\item To get a precise understanding of the notions involved, we have started
  a formalization in the Isabelle proof assistant
  Isabelle\footnote{\url{https://isabelle.in.tum.de/}}. The formalization of
  TypeDB can be found in Chapter~\ref{ch:isabelle_formalization}. 
\item Issues that come up during formalization will first be described in more
  detail in Section~\ref{sec:open_issues} and, upon resolution, migrate to
  Section~\ref{sec:resolved_issues} with a justification of the adopted
  solution.
\end{itemize}

Sources of the document and the formalizations are available on
Github\footnote{\url{https://github.com/Martin-Strecker/formal_typedb}}.


Our main \textbf{sources of information} are the following:

\begin{itemize}
\item the TypeDB documentation available on the
  web\footnote{\url{https://docs.vaticle.com/}},
  henceforth referenced as [DOC];
\item a video about knowledge
  graphs\footnote{\url{https://vaticle.com/use-cases/knowledge-graphs}},
  henceforth referenced as [KGV], or for a particular instant (minute, second)
  within this video as [KGV:min:sec] or simply [KGV:min]
\item the
  grammar\footnote{\url{https://github.com/vaticle/typeql/blob/master/grammar/TypeQL.g4}}
  referred to as [GR].
\end{itemize}

\paragraph{Acknowledgements:} The work has grown out of an internship by
Cécile Prieur, which has been co-supervised by Ralph Matthes. We gratefully
acknowledge their ideas and contributions. The internship's
Github repository is still available\footnote{\url{https://github.com/Martin-Strecker/stage_typedb}}. 



%%% Local Variables:
%%% mode: latex
%%% TeX-master: "main"
%%% coding: utf-8
%%% End:
